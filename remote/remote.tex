%-----------------------------------------------------------------------------
\section{Keeping it going...}

\subsection{I want to start a job up on a remote computer and make sure it keeps running
  even if I am disconnected}

{\tt nohup} is your friend here.  If you start you job with {\tt
  nohup}, then even if you close the terminal window, kill your {\tt
  ssh} session, etc., it will keep running.  Note that you should redirect
the output (including {\tt stderr})

\begin{verbatim}
nohup myjob &> out &
\end{verbatim}

For some commands, especially those run with MPI, you also want to redirect input,
such as:
\begin{verbatim}
nohup mpiexec -n 4 ./myjob &> out < /dev/null &
\end{verbatim}



\subsection{I want to be able to reconnect to a job that was running in
  a terminal that got disconnected}

GNU screen is the solution---this sets up a special terminal session that
can be detacted from your terminal (i.e., you can logout, close the window, etc.),
and then restore the session when you log back in.  For {\tt screen},
commands are entered by first typing {\tt \carat-a}, where ``{\tt \carat}'' is the
control key.

