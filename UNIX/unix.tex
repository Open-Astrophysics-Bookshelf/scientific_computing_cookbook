The standard (and recommended) UNIX shell is \bash.

setting variables

referencing variables

commandline arguments

backticks

\section{Loops}

\subsection{\em I want to loop over a set of JPEG images and convert them to PNG format}

The basic looping construct is the {\tt for} statement.  You can use this with wildcards to
loop over files.  For example, this will just print ({\tt echo}) the names of the JPEG
files in your directory:

\begin{lstlisting}[language={bash},upquote=true]
for i in *.jpg
do
   echo $i
done
\end{lstlisting}

You can type that line-by-line at the command prompt, or include it in
a text file (traditionally with the suffix `{\tt .sh}') and use the
{\tt source} command to execute the contents of the file.

To convert the images, we'll use the Imagemagick {\tt convert}
command---this is usually installed by default on most UNIX machines.
Another helpful command we'll use is {\tt basename}---this can extract
the basename of the filename without the extension.  Here's the
sequence to batch-convert the images:

\begin{lstlisting}[language={bash}]
for i in *.jpg
do
   convert $i `basename $i .jpg`.png
done
\end{lstlisting}







\section{Finding and replacing}



